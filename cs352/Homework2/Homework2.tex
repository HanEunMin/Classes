\documentclass[11pt,notitlepage]{article}
\usepackage{graphicx}
\usepackage{amsmath}            % adds more math symbols
\usepackage{amssymb}
\usepackage[usenames,dvipsnames]{color}
\usepackage{multicol}
\usepackage{listings}
\title{CS c352: Homework 2}
\author{Leo Przybylski\\
\texttt{przybyls@arizona.edu}}

\newcommand{\question}[2]{\textbf{#1.} #2}
\newcommand{\subquestion}[2]{\par\hspace{0.5cm} \textbf{#1)} #2}
\newenvironment{answer}{\endpar%

}

    % Give wider margins; gives more text per page.

\setlength{\topmargin}{0.00in}
\setlength{\textheight}{8.75in}
\setlength{\textwidth}{6.625in}
\setlength{\oddsidemargin}{0.0in}
\setlength{\evensidemargin}{0.0in}

\setlength{\parindent}{0.0cm}	% Don't indent the paragraphs
%\setlength{\parskip}{0.4cm}	% distance between paragraphs

\definecolor{ubergray}{RGB}{245,245,245}
\begin{document}
  \maketitle
  {\setlength{\baselineskip}%
           {0.0\baselineskip}
  \section*{Shell I/O and Piping}
  \hrulefill \par}
Task: Answer the following questions related to I/O redirection and piping:
\question{a}{I/O Redirection.}
\subquestion{i}{What does this command do? cat < oldfile > newfile Assume that oldfile exists
prior to the execution, but newfile does not.}

\begin{quote}
  redirects input from oldfile to the console and subsequently
  redirects that to newfile. newfile does not exist yet, so it will be created.
\end{quote}

\subquestion{ii}{Is it possible, on one command, to get Bash to redirect both stdout to one file and stderr
to a second file? If so, how? If not, why can’t Bash (or any shell) do
this?}

\begin{quote}
  Yes. \verb|outputsstderrandstdout.sh > some file 2> someotherfile|
\end{quote}

\subquestion{iii}{Is it possible, on one command, to use I/O redirection to place both stdout and stderr into
the same file? If so, how? If not, why can’t Bash (or any shell) do
this?}

\begin{quote}
  Yes. \verb|outputsstderrandstdout.sh > somefile 2>&1|
\end{quote}

\question{b}{Pipes.}
\subquestion{i}{Use ls to display the content of the directory /etc/X11 on lectura. How many top-level
subdirectories does that directory contain?}

\begin{lstlisting}[language=bash,basicstyle=\scriptsize,backgroundcolor=\color{ubergray},caption={trnapp-config.xml},frame=single,breaklines=true]
przybyls@lectura:~$ ls -l /etc/X11/ | grep -r '^d' | wc -l
12
przybyls@lectura:~$
\end{lstlisting}

\subquestion{ii}{In general, what does the du utility do?}

\begin{quote}
  'du' stands for disk usage. It traverses a path and returns the
  usage on disk that is taken up by that path.
\end{quote}

\subquestion{iii}{Use du to report the total size, in megabytes, of
/etc/X11 on lectura. What is its size?}

\begin{lstlisting}[language=bash,basicstyle=\scriptsize,backgroundcolor=\color{ubergray},caption={trnapp-config.xml},frame=single,breaklines=true]
przybyls@lectura:~$ du -sh /etc/X11
1.2M	/etc/X11
przybyls@lectura:~$ 
\end{lstlisting}

\subquestion{iv}{In general, what does the cut utility do?}

\begin{quote}
  It 'cuts' out fields from a string
\end{quote}

\subquestion{v}{Using a pipe, create a one-line command that uses both du and cut to list the names of
the top-level subdirectories in /etc/X11 on lectura. (Yes, your output will have a lot of
duplicates! We’ll address that shortly.) What is your command?}

\begin{lstlisting}[language=bash,basicstyle=\scriptsize,backgroundcolor=\color{ubergray},caption={trnapp-config.xml},frame=single,breaklines=true]
przybyls@lectura:~$ du -S /etc/X11/|cut -f2| cut -d / -f4
Xresources
Xreset.d
xkb
fonts
fonts
fonts
fonts
fonts
cursors
ko_KR.eucKR
ko_KR.eucKR
fluxbox
xinit
xinit
twm
ja_JP.eucJP
ja_JP.eucJP
Xsession.d
app-defaults
\end{lstlisting}

\subquestion{vi}{Identify a UNIX utility that can remove duplicate lines from output. Add this utility to
your answer from part (v) to remove the duplicates. Again, we’re looking for a one-line
command created with pipes. What is your command?}

\begin{lstlisting}[language=bash,basicstyle=\scriptsize,backgroundcolor=\color{ubergray},caption={trnapp-config.xml},frame=single,breaklines=true]
przybyls@lectura:~$ du -S /etc/X11/|cut -f2| cut -d / -f4|uniq
Xresources
Xreset.d
xkb
fonts
cursors
ko_KR.eucKR
fluxbox
xinit
twm
ja_JP.eucJP
Xsession.d
app-defaults
\end{lstlisting}

\subquestion{vii}{Augment your one-line command from part (vi) so that the output is in ascending order,
alphabetically, and state the resulting one-line command}

\begin{lstlisting}[language=bash,basicstyle=\scriptsize,backgroundcolor=\color{ubergray},caption={trnapp-config.xml},frame=single,breaklines=true]
przybyls@lectura:~$ du -S /etc/X11/|cut -f2| cut -d / -f4|uniq|sort

app-defaults
cursors
fluxbox
fonts
ja_JP.eucJP
ko_KR.eucKR
twm
xinit
xkb
Xreset.d
Xresources
Xsession.d
\end{lstlisting}

\newpage
  {\setlength{\baselineskip}%
           {0.0\baselineskip}
  \section*{Notes and Instructor Comments}
  \hrulefill \par}
\end{document}
