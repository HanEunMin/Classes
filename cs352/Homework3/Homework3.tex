\documentclass[11pt,notitlepage]{article}
\usepackage{graphicx}
\usepackage{amsmath}            % adds more math symbols
\usepackage{amssymb}
\usepackage[usenames,dvipsnames]{color}
\usepackage{multicol}
\usepackage{listings}
\title{CS c352: Homework 3}
\author{Leo Przybylski\\
\texttt{przybyls@arizona.edu}}

\newcommand{\question}[2]{\textbf{#1.} #2}
\newcommand{\subquestion}[2]{\par\hspace{0.5cm} \textbf{#1)} #2}
\newenvironment{answer}{\endpar%

}

    % Give wider margins; gives more text per page.

\setlength{\topmargin}{0.00in}
\setlength{\textheight}{8.75in}
\setlength{\textwidth}{6.625in}
\setlength{\oddsidemargin}{0.0in}
\setlength{\evensidemargin}{0.0in}

\setlength{\parindent}{0.0cm}	% Don't indent the paragraphs
%\setlength{\parskip}{0.4cm}	% distance between paragraphs

\definecolor{ubergray}{RGB}{245,245,245}
\begin{document}
  \maketitle
  {\setlength{\baselineskip}%
           {0.0\baselineskip}
  \section*{File Permissions}
  \hrulefill \par}
Task: Answer the following questions related to UNIX-style file and directory permissions.
\question{a}{What is your account’s current umask?}

\begin{lstlisting}[language=bash,basicstyle=\scriptsize,backgroundcolor=\color{ubergray},caption={trnapp-config.xml},frame=single,breaklines=true]
przybyls@lectura:~$ umask
0022
\end{lstlisting}

\question{b}{Assuming a default permission of 666 \texttt{(-rw-rw-rw-)} for ordinary files, what must your umask
be so that a newly-created file will have an initial permission of \texttt{246 (--w-r--rw-)}? If no such
umask can exist, explain why not.}

\begin{lstlisting}[language=bash,basicstyle=\scriptsize,backgroundcolor=\color{ubergray},caption={trnapp-config.xml},frame=single,breaklines=true]
przybyls@lectura:~$ umask 0420
przybyls@lectura:~$ touch blah
przybyls@lectura:~$ ls -l blah
--w-r--rw- 1 przybyls przybyls 0 2011-09-15 15:17 blah
\end{lstlisting}

\question{c}{Repeat (b), for an initial permission of 700
  \texttt{(-rwx------)} instead of 246.}

\begin{quote}
  I think this is impossible. Here's why. If the default permission is
  666, then the bit arrangement is:

\begin{tabular}{ccc}
  110 & 110 & 110 
\end{tabular}

Many think that the way the permission mask works is that the mask is
applied to the default permission with a logical or (\^). That is not
true though. If that were true, it would be easy to find the umask by
using the logical or

\begin{tabular}{cccc}
  & 110 & 110 & 110 \\
\verb|^| & 111 & 000 & 000 \\
\hline \\
& 001 & 110 & 110
\end{tabular}

But that does not work. See below:

\begin{lstlisting}[language=bash,basicstyle=\scriptsize,backgroundcolor=\color{ubergray},caption={trnapp-config.xml},frame=single,breaklines=true]
przybyls@lectura:~$ umask 177
przybyls@lectura:~$ rm blah 
przybyls@lectura:~$ touch blah
przybyls@lectura:~$ ls -l blah
-rw------- 1 przybyls przybyls 0 2011-09-15 15:27 blah
\end{lstlisting}

So why does this not work? The reason is that it's not a logical or
(\verb|^|). It's actually a bitwise \verb|& ~|. Which would lead to no
viable mask that can be applied to 666 default permission to provide 700.

\end{quote}

\question{On lectura, if you change to your home directory with cd /home/NETID (replacing ‘NETID’
with your NETID, of course), it’ll work — but your home directory isn’t really at /home/NETID.
What is the true path to your home directory?}

\begin{lstlisting}[language=bash,basicstyle=\scriptsize,backgroundcolor=\color{ubergray},caption={trnapp-config.xml},frame=single,breaklines=true]
przybyls@lectura:~$ ls -l $HOME
lrwxrwxrwx 1 root wheel 15 2011-09-01 09:39 /home/przybyls ->
/p2/hp/przybyls
\end{lstlisting}

\question{What is the difference between chmod +x foo and chmod u+x
  foo?}

\begin{quote}
\verb|chmod +x foo| will grant \verb|x| to all, groups, and the
user. 

\verb|chmod u+x foo| will grant \verb|x| to just the user/owner of foo.
\end{quote}


\newpage
  {\setlength{\baselineskip}%
           {0.0\baselineskip}
  \section*{Notes and Instructor Comments}
  \hrulefill \par}
\end{document}
