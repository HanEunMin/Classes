\documentclass[11pt,notitlepage]{article}
\usepackage{graphicx}
\usepackage{amsmath}            % adds more math symbols
\usepackage{amssymb}
\usepackage[usenames,dvipsnames]{color}
\usepackage{multicol}
\usepackage{listings}
\title{LING-300: Homework 6}
\author{Leo Przybylski\\
\texttt{przybyls@arizona.edu}}

\newcommand{\question}[2]{\textbf{#1.} #2}
\newcommand{\subquestion}[2]{\par\hspace{0.5cm} \textbf{#1)} #2}
\newenvironment{answer}{\endpar%

}

    % Give wider margins; gives more text per page.

\setlength{\topmargin}{0.00in}
\setlength{\textheight}{8.75in}
\setlength{\textwidth}{6.625in}
\setlength{\oddsidemargin}{0.0in}
\setlength{\evensidemargin}{0.0in}

\setlength{\parindent}{0.0cm}	% Don't indent the paragraphs
%\setlength{\parskip}{0.4cm}	% distance between paragraphs

\definecolor{ubergray}{RGB}{245,245,245}
\begin{document}
  \maketitle
  {\setlength{\baselineskip}%
           {0.0\baselineskip}
  \section*{Carnie Chapter 6}
  \hrulefill \par}

\question{1}{COMPLEMENTS VS. ADJUNCTS in NPs}
\begin{quote}
[Application of Skills; Basic]
Using the tests you have been given (reordering, adjacency,
conjunction of likes, one-replacement) determine whether the PPs in
the following NPs are complements or adjuncts; give the examples that
you used in constructing your tests. Some of the NPs have multiple
PPs, be sure to answer the ques- tion for every PP in the NP.
\end{quote}
\subquestion{a}{A container [of flour]}
\begin{quote}
``of flour'' is a complement. It passes the adjacency test and the
heuristic test of prepositions using the ``of'' preposition. It also
matches the rule stating that the complement is a sister to a head.
\end{quote}

\subquestion{b}{A container [with a glass lid]}
\begin{quote}
  ``with a glass'' lid is a complement. It passes the adjacency test and the
heuristic test of prepositions using the ``of'' preposition. It also
matches the rule stating that the complement is a sister to a head.
\end{quote}

\subquestion{c}{The collection [of figurines] [in the window]}
\begin{quote}

 ``of figurines'' is a complement. It passes the adjacency test and the
heuristic test of prepositions using the ``of'' preposition. It also
matches the rule stating that the complement is a sister to a head.

``in the window'' is an adjunct. It passes the rule stating that an
adjunct is the sister to a bar. In this case, the bar ``of figurines'' is the complement.
  
\end{quote}

\subquestion{d}{The statue [of Napoleon] [on the corner]}
\begin{quote}
  ``of Napoleon'' is a complement. ``on the corner'' is an adjunct.
\end{quote}

\subquestion{e}{Every window [in the building] [with a broken pane]}
\begin{quote}
  ``in the building'' is an adjunct. ``with a broken pane'' is an adjunct.
\end{quote}


\question{6}{PARAMETERS}
\begin{quote}[Data Analysis; Basic to Intermediate]
Go back to the foreign language problems from chapters 3 and 4,
(Hiaki, Irish, Bambara, Hixkaryana, Swedish, Dutch, Tzotzil) and see
if you can de- termine the parameter settings for these languages. You
may not be able to determine all the settings for each
language. (Suggestion: put your answer in a table like the one
below. English is done for you as an example.) Assume the following:
Determiners are typical examples of specifiers, Adjectives and many
PPs (although not all) are adjuncts. “of” PPs and direct objects are
complements. Be sure to check the complement/adjunct relation in all
cate- gories (N, Adj, Adv, V, P etc.) if you can.
\end{quote}

\question{7}{TREES}
\begin{quote}
[Application of Skills; Basic to Advanced]
Draw the X-bar theoretic trees for the following sentences. Treat
possessive NPs like Héloïse's as specifiers. Several of the sentences
are ambiguous; draw only one tree, but indicate using a paraphrase (or
paraphrases) which meaning you intend by your tree.
\end{quote}
a) Abelard wrote a volume of poems in Latin for Héloïse.
b) Armadillos from New York often destroy old pillowcases with their
snouts. (NB: assume "their" is a determiner)
c) People with boxes of old clothes lined up behind the door of the building
with the leaky roof.
d) That automobile factories abound in Michigan worries me greatly.
e) No-one understands that phrase structure rules explain the little under-
stood phenomenon of the infinite length of sentences.
f) My Favorite language is a language with simple morphology and compli-
cated syntax.
g) Ivan got a noogie on Wednesday from the disgruntled students of pho-
nology from Michigan.
h) The collection of syntax articles with the red cover bores students of syn-
tax in Tucson
i) The red volume of obscene verse from Italy shocked the puritan soul of
the minister with the beard quite thoroughly yesterday.
j) The biggest man in the room said that John danced an Irish jig from
County Kerry to County Tipperary on Thursday.

\newpage
  {\setlength{\baselineskip}%
           {0.0\baselineskip}
  \section*{Notes and Instructor Comments}
  \hrulefill \par}
\end{document}
