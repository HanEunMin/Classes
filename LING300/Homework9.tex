\documentclass[11pt,notitlepage]{article}
\usepackage{graphicx}
\usepackage{amsmath}            % adds more math symbols
\usepackage{amssymb}
\usepackage[usenames,dvipsnames]{color}
\usepackage{multicol}
\usepackage{listings}
\title{LING-300: Homework 9}
\author{Leo Przybylski\\
\texttt{przybyls@arizona.edu}}

\newcommand{\question}[2]{\textbf{#1.} #2}
\newcommand{\subquestion}[2]{\par\hspace{0.5cm} \textbf{#1)} #2}
\newenvironment{answer}{\endpar%

}

    % Give wider margins; gives more text per page.

\setlength{\topmargin}{0.00in}
\setlength{\textheight}{8.75in}
\setlength{\textwidth}{6.625in}
\setlength{\oddsidemargin}{0.0in}
\setlength{\evensidemargin}{0.0in}

\setlength{\parindent}{0.0cm}	% Don't indent the paragraphs
%\setlength{\parskip}{0.4cm}	% distance between paragraphs

\definecolor{ubergray}{RGB}{245,245,245}
\begin{document}
  \maketitle
  {\setlength{\baselineskip}%
           {0.0\baselineskip}
  \section*{Carnie Chapter 8}
  \hrulefill \par}

\question{2}{WARLPIRI^3}
\begin{quote}[Data Analysis; Basic]
Consider the following data from Warlpiri:
\end{quote}

\subquestion{a}{Lungkarda ka ngulya-ngka nguna-mi. bluetongue AUX
  burrow-A lie-NON.PAST “The bluetongue skink is lying in the
  burrow.”}

\begin{description}
  \item [-ngka] in this context means ``in'' or ``in the burrow''.
\end{description}

\subquestion{b}{Nantuwu ka karru-kurra parnka-mi.}
horse AUX creek-B
“The horse is running to the creek.”
run-NON.PAST

\begin{description}
  \item [-kurra] in this context means ``to'' or ``toward'' like ``to
    the creek''.
\end{description}

\subquestion{c}{Karli ka pirli-ngirli wanti-mi. boomerang AUX stone-C falngkal-NON.PAST “The boomerang is falling from the stone.”}

\begin{description}
  \item [-ngirli] in this context means ``from'' or ``from the stone''.
\end{description}

\subquestion{d}{Kurdu-ngku ka-jana pirli yurutu-wana yirra-rni. child-D AUX stone road-E put.NON.PAST “The child is putting stones along the road.”}

\begin{description}
  \item [-wana] in this context means ``along'' or ``along the road''.
\end{description}


What is the meaning of each of the affixes (suffixes) glossed with -A,
-B, -C, -D, and -E. Can you relate these suffixes to thematic
relations? Which ones?

\begin{quote}
  They all seem thematic. They all are prepositions. They all suffix a
  noun even though they're modifying a verb.
\end{quote}


\question{3}{THETA GRIDS}
\begin{quote}
[Data Analysis; Basic]
For each of the sentences below identify each of the predicates
(including non-verbal predicates like is likely). Provide the theta
grid for each. Don’t forget: include only arguments in the theta grid;
DPs and PPs that are ad- juncts are not included. Index each DP, PP,
CP argument with the theta role it takes. Assume that there are two
different verbs give (each with their own theta grids) to account for
(c) and (d); two different verbs eat (each with their own theta grids
for (e) and (f); and two different verbs ask for (i) and (j).\end{quote}

\subquestion{d}{Mercedes gave a test to the students in the lecture
  hall.}

\subquestion{e}{Pangur ate a cat treat.}

\subquestion{f}{Susan ate yesterday at the restaurant.}

\subquestion{g}{Gwen saw a fire truck.}

\subquestion{h}{Gwen looked at a fire truck.}

\question{5}{THETA CRITERION}
\begin{quote}
[Data Analysis; Intermediate]
Show how each of the following sentences are violations of the theta
criteria) *Rosemary hates.
b) *Jennie smiled the breadbox.
c) *Traci gave the whale.
d) *Traci gave a jawbreaker.a) *Rosemary hates.
b) *Jennie smiled the breadbox.
c) *Traci gave the whale.
d) *Traci gave a jawbreaker.rion. Use theta grids to explain your answers.
\end{quote}

\question{a}{*Rosemary hates.}

\question{b}{*Jennie smiled the breadbox.}

\question{c}{*Traci gave the whale.}

\question{d}{*Traci gave a jawbreaker.}

\newpage
  {\setlength{\baselineskip}%
           {0.0\baselineskip}
  \section*{Notes and Instructor Comments}
  \hrulefill \par}
\end{document}
