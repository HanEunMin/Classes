\documentclass[11pt,notitlepage]{article}
\usepackage{graphicx}
\usepackage{amsmath}            % adds more math symbols
\usepackage{amssymb}
\usepackage[usenames,dvipsnames]{color}
\usepackage{multicol}
\usepackage{listings}
\title{LING-300: Homework 1}
\author{Leo Przybylski\\
\texttt{przybyls@arizona.edu}}

\newcommand{\question}[2]{\textbf{#1.} #2}
\newcommand{\subquestion}[2]{\par\hspace{0.5cm} \textbf{#1)} #2}
\newenvironment{answer}{\endpar%

}

    % Give wider margins; gives more text per page.

\setlength{\topmargin}{0.00in}
\setlength{\textheight}{8.75in}
\setlength{\textwidth}{6.625in}
\setlength{\oddsidemargin}{0.0in}
\setlength{\evensidemargin}{0.0in}

\setlength{\parindent}{0.0cm}	% Don't indent the paragraphs
%\setlength{\parskip}{0.4cm}	% distance between paragraphs

\definecolor{ubergray}{RGB}{245,245,245}
\begin{document}
  \maketitle
  {\setlength{\baselineskip}%
           {0.0\baselineskip}
  \section*{Carnie Chapter 2}
  \hrulefill \par}

\question{1}{PART OF SPEECH 19}
\begin{quote}
[Application of Skills; Basic]
Identify the main parts of speech (i.e., Nouns, Verbs,
Adjectives/Adverbs, and Prepositions) in the following
sentences. Treat hyphenated words as single words:
\end{quote}

\subquestion{a}{The old rusty pot-belly stove has been replaced.}
\begin{quote}
  The (D) old (Adj) rusty (Adj) pot-belly (Adj) stove (N) has (T) been
  (T) replaced (V).
\end{quote}

\subquestion{b}{The red-haired assistant put the vital documents
  through the new efficient shredder.}

\begin{quote}
  The (D) red-haired (Adj) assistant (N) put (V) the (D) vital (Adj)
  documents (N) through (P) the (D) new (Adj) efficient (Adj) shredder (N).
\end{quote}

\subquestion{c}{The large evil leathery alligator complained to his aging keeper about his
extremely unattractive description.}

\begin{quote}
  The (D) large (Adj) evil (Adj) leathery (Adj) alligator (V)
  complained (V) to (P) his (D) aging (Adj) keeper (N) about (P) his
  (N) extremely (Adj) unattractive (Adj) description (N).
\end{quote}

\subquestion{d}{I’ve just eaten the last piece of chocolate cake.}

\question{2}{NOOTKA}
\begin{quote}
[Application of Skills; Intermediate]
Consider the following data from Nootka (data from Sapir and Swadesh 1939), a language spoken in British Columbia, Canada and answer the questions that follow the grey text box.
a) Mamu:k-ma qu:ʔas-ʔi. working-PRES man-DEF “The man is working.”
b) Qu:ʔas-ma mamu:k-ʔi. man-PRES working-DEF “The working one is a man.”
(The : mark indicates a long vowel. ʔ is a glottal stop. PRES in the second line means “present tense,” DEF means “definite determiner” (the).)
Questions about Nootka:
\end{quote}

\subquestion{1}{In sentence a, is Qu:ʔas functioning as a verb or a
  noun?}

\begin{quote}
  Noun
\end{quote}

\subquestion{2}{In sentence a, is Mamu:k functioning as a verb or a
  noun?}

\begin{quote}
  Verb
\end{quote}

\subquestion{3}{In sentence b, is Qu:ʔas a verb or a noun?}

\begin{quote}
Verb
\end{quote}

\subquestion{4}In sentence b, is Mamu:k a verb or a noun?

\begin{quote}
  Noun
\end{quote}

\subquestion{5}{What criteria did you use to tell what is a noun in Nootka and what is a
verb?}

\begin{quote}
  Tense and the phrase structure. Nootka sentences begin with a verb,
  so the verb is always first. Also, the first word happens to have
  tense associated with it which would make it a verb.
\end{quote}

\subquestion{6}{How does this data support the idea that there are no semantic criteria
involved in determining the part of speech?}

\begin{quote}
  When a word can switch between verb or noun freely, you cannot
  derive parts of speech from the semantically because it could be
  either or.
\end{quote}

\question{6}{SUBCATEGORIES OF NOUNS}
\begin{quote}[Application of Knowledge; Basic]
For each of the nouns below put a + sign in the box under the features
that they have. Note that some nouns might have a plus value for more
than one feature. The first one is done for you. Do not mark the minus
(–) values, or the values for which the word is not specified; mark
only the plus values!\end{quote}

\begin{tabular}{|l|c|c|c|c|c|}
  \hline \\
  Noun & Plural & Count & Proper & Pronoun & Anaphor\\
  \hline \\
  Cats & + & + & & & \\
  \hline \\
  Milk & & & & & \\
  \hline \\
  New York & & & + & & \\
  \hline \\
  They & + & & & + & \\
  \hline \\
  People & + & + & & & \\
  \hline \\
  Language & & & & & \\
  \hline \\
  Printer & & & & & \\
  \hline \\
  Himself & & & & & + \\ 
  \hline \\
  Wind & & & & & \\
  \hline \\
  Lightbulb & & & & & \\
  \hline
\end{tabular}

\question{7}{SUBCATEGORIES OF VERBS}
\begin{quote}
[Application of Knowledge; intermediate]
For each of the verbs below, list whether they are intransitive,
transitive or ditransitive and list which features they take (see the
list in (32) as an exam- ple). In some cases they may allow more than
one feature. E.g., the verb eat is both $\left[NP \_\_ NP\right]$ and $\left[NP \_\_\_ \right]$. Give
an example for each feature:
\begin{quote}
  \emph{spray, sleep, escape, throw, wipe, say, think, grudge, thank, pour,
    send, promise, kiss, arrive}
\end{quote}

\begin{description}
\item [spray] transitive (NP\_\_NP)
\item [sleep] intransitive (NP\_\_) and transitive (NP\_\_NP)
\item [escape] intransitive (NP\_\_), transitive (NP\_\_NP), and transitive (NP\_\_PP)
\item [throw] transitive (NP\_\_NP) and ditransitive (NP\_\_NP PP)
\item [wipe] transitive (NP\_\_NP) and ditransitive (NP\_\_NP PP)
\item [say] transitive (NP\_\_{NP/CP}) and ditransitive (NP\_\_NP PP)
\item [think] transitive (NP\_\_{NP/CP/PP}) 
\item [grudge] intransitive (NP\_\_)
\item [thank] transitive (NP\_\_{NP/CP}) and ditransitive (NP\_\_NP PP)
\item [pour] transitive (NP\_\_NP) and ditransitive (NP\_\_NP PP)
\item [send] transitive (NP\_\_NP) and ditransitive (NP\_\_NP PP)
\item [promise] transitive (NP\_\_NP) and ditransitive (NP\_\_NP PP)
\item [kiss] transitive (NP\_\_NP) and ditransitive (NP\_\_NP PP)
\item [arrive] intransitive (NP\_\_) and transitive (NP\_\_{NP/PP})
\end{description}

\end{quote}

\newpage
  {\setlength{\baselineskip}%
           {0.0\baselineskip}
  \section*{Notes and Instructor Comments}
  \hrulefill \par}
\end{document}
