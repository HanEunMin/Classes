\documentclass[11pt]{report}
\usepackage{graphicx}
\usepackage{amsmath}            % adds more math symbols
\usepackage{amssymb}
\usepackage[usenames,dvipsnames]{color}
\usepackage{multicol}
\usepackage{listings}
\title{Word Order Project}
\author{Leo Przybylski\\
\texttt{przybyls@arizona.edu}}
\date{}

\newcommand{\question}[2]{\textbf{#1.} #2}
\newcommand{\subquestion}[2]{\par\hspace{0.5cm} \textbf{#1)} #2}
\newenvironment{answer}{\endpar%

}

    % Give wider margins; gives more text per page.

\setlength{\topmargin}{0.00in}
\setlength{\textheight}{8.75in}
\setlength{\textwidth}{6.625in}
\setlength{\oddsidemargin}{0.0in}
\setlength{\evensidemargin}{0.0in}

\setlength{\parindent}{0.0cm}	% Don't indent the paragraphs
%\setlength{\parskip}{0.4cm}	% distance between paragraphs

\definecolor{ubergray}{RGB}{245,245,245}
\begin{document}
\maketitle
\abstract{A word order project for LING-300 instructed by Jeff Punske
  at the University of Arizona. This paper explains the word order for
the Korean spoken and written language, Hangul as well as the research
required to arrive at that conclusion. Examples, are given as X' bar
trees to illustrate proofs and experiments used to determine the word
order. The case for different types of movement is explicitly
given. The goal is to first determine the word order, then provide X'
bar rules for it.}

\section{Introduction}
The word order for Korean/Hangul may look on the surface to be a
subject-object-verb (SOV) word order language; however, in exceptional
cases like tense or wh- movement, it is possible that it is not
actually so. In order to be a true SOV word order language, Korean
must be SOV in all cases. This includes the exceptional ones. I will show that Korea actually is an SOV language by contradicting
it as any other word order, and that this is so especially in cases of
movement.

\section{Hangul as a Subject-Object-Verb Word Order Language}
Lanugages such as Irish (VSO) or English (SVO) have distinguishable
rules about movement and phrase structure that differ from
Hangul. Before moving onto rules and movement, it should be known that
unlike the aforementioned languages, Hangul is a head-final
language. This means that the head appears on the right in a
tree. Trees tend to branch out away from the head, so a PST for Hangul
will typically branch left instead of right because of its head-final
nature. This is important also because rules will appear to be
inverted right-to-left, but they are not. It is just that Hangul is
head-final.

Also, as a result of being head-final, complements and adjuncts will
be on the left of whatever they modify. Since complements are
head-adjacent, a complement is then directly to the left of whatever
it is modifying.

\subsection{X-bar Rules}
I have determined that the X-bar rules for Hangul are as follows:

\begin{tabular}{lcr}
  CP& $\rightarrow$ & C' \\
  C' & $\rightarrow$ & VP (C) (VP)\\
  VP & $\rightarrow$ & (DP) V' \\ 
  DP & $\rightarrow$ & D' (DP) \\
  D' & $\rightarrow$ & (D) NP (D) \\
  NP & $\rightarrow$ & N' PP \\
  N' & $\rightarrow$ & (AdjP) (N) \\ 
  AdjP & $\rightarrow$ & AdjP' (AdjP) \\ 
  AdjP' & $\rightarrow$ & AdjP \\ 
  PP & $\rightarrow$ & DP P \\ 
  V' & $\rightarrow$ & (AdvP+) V (TP) (CP) \\ 
  AdvP & $\rightarrow$ & AdvP' (AdjP+) \\ 
  AdvP' & $\rightarrow$ & AdvP \\ 
  TP & $\rightarrow$ & T' \\ 
  T' & $\rightarrow$ & T   
\end{tabular}

Since Hangul is head-final and the tree branches left from right,
then the verb phrase becomes the predominant next to the
complementizer. The Determiner is then a daughter of the Verb. The
Determiner is also optional. The reason for this is that the subject
in Hangul is optional. It is implied. As a result, context becomes
very important and so a lot of the movement (as I describe later) is
discourse driven.

Take for example, the phrase \emph{John is likely to leave.} In Hangul
even John can be implied, so the sentence can possibly be, ``tteonal
ganeungseong-i nopseubnida'' or ``likely to leave''. This is perfectly
grammatical in Korean since the subject is determined by the context.

\subsection{Tense}
Hangul tense modifiers are entirely on the verb. This makes tense
movement very simple. Observe the following sentences:

\emph{Did you see the dog?}

\begin{tabular}{llll}
You & dog & saw? \\
dangsin-eun & gaeleul & bwass-eo?
\end{tabular}

\includegraphics[scale=0.75]{Diagrams/WordOrder2.png}

Tense is interesting here because it follows the verb and is moved up
in a ``raising'' technique. 

\emph{``Are you going?''}

\begin{tabular}{llll}
Going? \\
ganeungeoya?
\end{tabular}

\includegraphics[scale=0.75]{Diagrams/WordOrder3.png}

This is interesting because there is movement, but it is from the
verb to the noun and then the tense takes over as the new
verb. ``geoya'' implies the intention of doing something and
``ganeun'' can be considered a separate phrase that makes the verb
``ga'' (to go) into a noun.


\subsection{Binding Theory}
Pronouns, r-expressions, and anaphors are sparse in Hangul. Pluralized
pronouns are entirely absent as well as anaphors. Just as in other
languages though, pronouns must be free within the binding domain. It
is rare for pronouns to be used due to the emphasis on
honorifics. Pronouns are devoid of honorifics, and so they are hardly
ever used. As a result, concepts of accusative and nominative case are
more difficult to discover.

\subsection{Movement}

\subsubsection{Wh- Movement}
There is no wh- movement in Hangul. In regular wh- movement, an object
is moved up to the head of the complementizer phrase. Hangul does not
require this. Instead, the object simply stays. Let us examine, the
sentence \emph{Whose book is this?}.

\begin{tabular}{llll}
this & book & whose & to have \\
i & chaeg-eun & nugu &ibnikka?
\end{tabular}

\includegraphics[scale=0.75]{Diagrams/WordOrder1.png}

You can see from sentence that the subject ``book'' starts the
sentence with ``whose'' as the object coming later. No movement is
necessary to place ``whose'' within a complementizer.

\subsubsection{T to C Movement}
There is not T to C movement in Hangul. See the previously illustrated sentence
\emph{Did you see the dog?}

\subsubsection{V to N Movement}
Something new. The verb becomes the noun in cases where tense takes
over the verb.

\subsection{DP Movement}
``I am busy doing something''

\begin{tabular}{llll}
I & busy & doing something &  \\
naneun & bappeun& mwongaleulhagoissneun& jung-iya
\end{tabular}

\includegraphics[scale=0.75]{Diagrams/WordOrder4.png}

The above shows that the progressive form of a verb requires V to N
Movement, but in this case ``am'' is the main verb which requires D to
V movement. The issue here is that there is a subject and verb
together ``I am'', but the object comes later. This requires special
movement to allow for a (SOV) word order. In this case, it is more of
a copy than a move because ``I'' stays firmly in the subject position.

\subsection{V to T Movement}
Because Hangul uses V to N movement for progressive forms and all
other tense related movement is from T to V, it is difficult to show
if there is any possibility of T to V movement.


\section{Conclusion}
There is no evidence that shows Hangul is capable of moving a NP or VP
in any order where the outcome would be anything but (SOV). Even when
moving NP to the VP, the subject still is firmly rooted at the
beginning of the sentence.


\begin{thebibliography}{9}
  \bibitem{Lee2000} King, Ross and Yeon, Jae-Hoon 2000. Elementary Korean
  \bibitem{Yeon2000} Cho, Young-Mee, Lee, Hyo Sang, Schulz, Carol,
    Sohn, Ho-Mi, Sohn, Sung-Ock 200. Integrated Korean
  \bibitem{lang1} Jurafsky, Daniel, Martin, James H., Jurafsky, Dan
    2008 (2nd ed.) Speech and Language Processing. An Introduction to Natural Language Processing, Computational Linguistics and Speech Recognition
\end{thebibliography}
\end{document}
